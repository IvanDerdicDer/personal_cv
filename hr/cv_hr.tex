\documentclass[10pt, letterpaper]{article}

% Packages:
\usepackage[
    ignoreheadfoot, % set margins without considering header and footer
    top=2 cm, % seperation between body and page edge from the top
    bottom=2 cm, % seperation between body and page edge from the bottom
    left=2 cm, % seperation between body and page edge from the left
    right=2 cm, % seperation between body and page edge from the right
    footskip=1.0 cm, % seperation between body and footer
    % showframe % for debugging 
]{geometry} % for adjusting page geometry
\usepackage{titlesec} % for customizing section titles
\usepackage{tabularx} % for making tables with fixed width columns
\usepackage{array} % tabularx requires this
\usepackage[dvipsnames]{xcolor} % for coloring text
\definecolor{primaryColor}{RGB}{0, 0, 0} % define primary color
\usepackage{enumitem} % for customizing lists
\usepackage{fontawesome5} % for using icons
\usepackage{amsmath} % for math
\usepackage[
    pdftitle={John Doe's CV},
    pdfauthor={John Doe},
    pdfcreator={LaTeX with RenderCV},
    colorlinks=true,
    urlcolor=primaryColor
]{hyperref} % for links, metadata and bookmarks
\usepackage[pscoord]{eso-pic} % for floating text on the page
\usepackage{calc} % for calculating lengths
\usepackage{bookmark} % for bookmarks
\usepackage{lastpage} % for getting the total number of pages
\usepackage{changepage} % for one column entries (adjustwidth environment)
\usepackage{paracol} % for two and three column entries
\usepackage{ifthen} % for conditional statements
\usepackage{needspace} % for avoiding page brake right after the section title
\usepackage{iftex} % check if engine is pdflatex, xetex or luatex

% Ensure that generate pdf is machine readable/ATS parsable:
\ifPDFTeX
    \input{glyphtounicode}
    \pdfgentounicode=1
    \usepackage[T1]{fontenc}
    \usepackage[utf8]{inputenc}
    \usepackage{lmodern}
\fi

\usepackage{charter}

% Some settings:
\raggedright
\AtBeginEnvironment{adjustwidth}{\partopsep0pt} % remove space before adjustwidth environment
\pagestyle{empty} % no header or footer
\setcounter{secnumdepth}{0} % no section numbering
\setlength{\parindent}{0pt} % no indentation
\setlength{\topskip}{0pt} % no top skip
\setlength{\columnsep}{0.15cm} % set column seperation
\pagenumbering{gobble} % no page numbering

\titleformat{\section}{\needspace{4\baselineskip}\bfseries\large}{}{0pt}{}[\vspace{1pt}\titlerule]

\titlespacing{\section}{
    % left space:
    -1pt
}{
    % top space:
    0.3 cm
}{
    % bottom space:
    0.2 cm
} % section title spacing

\renewcommand\labelitemi{$\vcenter{\hbox{\small$\bullet$}}$} % custom bullet points
\newenvironment{highlights}{
    \begin{itemize}[
        topsep=0.10 cm,
        parsep=0.10 cm,
        partopsep=0pt,
        itemsep=0pt,
        leftmargin=0 cm + 10pt
    ]
}{
    \end{itemize}
} % new environment for highlights


\newenvironment{highlightsforbulletentries}{
    \begin{itemize}[
        topsep=0.10 cm,
        parsep=0.10 cm,
        partopsep=0pt,
        itemsep=0pt,
        leftmargin=10pt
    ]
}{
    \end{itemize}
} % new environment for highlights for bullet entries

\newenvironment{onecolentry}{
    \begin{adjustwidth}{
        0 cm + 0.00001 cm
    }{
        0 cm + 0.00001 cm
    }
}{
    \end{adjustwidth}
} % new environment for one column entries

\newenvironment{twocolentry}[2][]{
    \onecolentry
    \def\secondColumn{#2}
    \setcolumnwidth{\fill, 4.5 cm}
    \begin{paracol}{2}
}{
    \switchcolumn \raggedleft \secondColumn
    \end{paracol}
    \endonecolentry
} % new environment for two column entries

\newenvironment{threecolentry}[3][]{
    \onecolentry
    \def\thirdColumn{#3}
    \setcolumnwidth{, \fill, 4.5 cm}
    \begin{paracol}{3}
    {\raggedright #2} \switchcolumn
}{
    \switchcolumn \raggedleft \thirdColumn
    \end{paracol}
    \endonecolentry
} % new environment for three column entries

\newenvironment{header}{
    \setlength{\topsep}{0pt}\par\kern\topsep\centering\linespread{1.5}
}{
    \par\kern\topsep
} % new environment for the header

\newcommand{\placelastupdatedtext}{% \placetextbox{}{}{}
  \AddToShipoutPictureFG*{% Add  to current page foreground
    \put(
        \LenToUnit{\paperwidth-2 cm-0 cm+0.05cm},
        \LenToUnit{\paperheight-1.0 cm}
    ){\vtop{{\null}\makebox[0pt][c]{
        \small\color{gray}\textit{Last updated in September 2024}\hspace{\widthof{Last updated in September 2024}}
    }}}%
  }%
}%

% save the original href command in a new command:
\let\hrefWithoutArrow\href

% new command for external links:


\begin{document}
    \newcommand{\AND}{\unskip
        \cleaders\copy\ANDbox\hskip\wd\ANDbox
        \ignorespaces
    }
    \newsavebox\ANDbox
    \sbox\ANDbox{$|$}

    \begin{header}
        \fontsize{25 pt}{25 pt}\selectfont Ivan Derdić

        \vspace{5 pt}

        \normalsize
        \mbox{Zagreb}%
        \kern 5.0 pt%
        \AND%
        \kern 5.0 pt%
        \mbox{\hrefWithoutArrow{mailto:ivan.derdic@proton.me}{ivan.derdic@proton.me}}%
        \kern 5.0 pt%
        \AND%
        \kern 5.0 pt%
        \mbox{\hrefWithoutArrow{tel:+385-91-933-1914}{+385 91 933 1914}}%
        \kern 5.0 pt%
        \AND%
        \kern 5.0 pt%
        \mbox{\hrefWithoutArrow{https://github.com/IvanDerdicDer}{github.com/IvanDerdicDer}}%
    \end{header}

    \vspace{5 pt - 0.3 cm}


    \section{Iskustvo}



        
        \begin{twocolentry}{
            Prosinac 2021 – Srpanj 2024
        }
            \textbf{Podatkovni inženjer}, Bonsai.TECH -- Zagreb, HR\end{twocolentry}

        \vspace{0.10 cm}
        \begin{onecolentry}
            \begin{highlights}
                \item Pilot projekt za Švedsku tvrtku koja se bavi posredovanjem nekretnina - izrada sustava za sinkroniziranje dva izvora podataka
                \item Kreiranje i održavanje podatkovne platforme za Starbucks - podatkovna platforma za objedinjavanje i analizu podataka o kupnjama i programa vrijednosti
                \item Obrada Oracle BIAPI podatka - sustav za obradu podataka o kupnjama u skoro pa stvarnom vremenu koji se dohvaćaju s Oracle BIAPI sustava
            \end{highlights}
        \end{onecolentry}


        \vspace{0.2 cm}

        \begin{twocolentry}{
            Rujan 2024 – Trenutačno
        }
            \textbf{Podatkovni inženjer}, Span -- Zagreb, HR\end{twocolentry}

        \vspace{0.10 cm}
        \begin{onecolentry}
            \begin{highlights}
                \item Održavanje i nadograđivanje sustava za obradu podataka o programu vrijednosti
                \item Izrada sustava za anonimizaciju podataka njihovim uklanjanjem
            \end{highlights}
        \end{onecolentry}


    \section{Projekti}

        \begin{onecolentry}
            \textbf{UnitType}
            \vspace{0.10 cm}
            \begin{highlights}
                \item Program za brzo pisanje korištenjem klasičnih tipkovnica. Program prati koje tipke su pritisnute te ako je važeća kombinacija pritisnuta (unit) dolazi do proširenja na cijelu riječ.
                \item Korišteni alati: Rust
            \end{highlights}
        \end{onecolentry}


        \vspace{0.2 cm}

        \begin{onecolentry}
            \href{https://github.com/IvanDerdicDer/Acimdes}{\textbf{Acimdes}}
            \vspace{0.10 cm}
            \begin{highlights}
                \item Program u komandnoj liniji za igranje kartaške igre sedmica
                \item Korišteni alati: Python
            \end{highlights}
        \end{onecolentry}


    \section{Obrazovanje}



        \begin{twocolentry}{
            Rujanj 2022 – Srpanj 2024
        }
            \textbf{Fakultet elektrotehnike i računarstva}, magistar inženjer računarstva\end{twocolentry}

        \vspace{0.10 cm}
        \begin{onecolentry}
            \begin{highlights}
                \item \textbf{Diplomski rad:} "Sustav za obradu polustrukturiranih tekstnih datoteka poslovnih transakcijskih podataka" - cilj projekta je bio implementirati različite načine obrade polustrukturiranih tekstnih datoteka i usporediti njihove vremenske performance
                \item \textbf{Kolegiji:} Napredne baze podataka, Vizualizacija podataka, Sustavi baza podataka, Paralelno programiranje, Upravljanje podacima
            \end{highlights}
        \end{onecolentry}    


        \begin{twocolentry}{
            Rujanj 2019 – Srpanj 2022
        }
            \textbf{Fakultet elektrotehnike i računarstva}, sveučilišni prvostupnik inženjer računarstva\end{twocolentry}

        \vspace{0.10 cm}
        \begin{onecolentry}
            \begin{highlights}
                \item Rad na projektu "FRISK V procesorska arhitektura" - cilj projekta je bio izraditi simulator RISC V procesorske arhitekture za potrebe kolegija "Arhitektura računala"
                \item \textbf{Završni rad:} "Sustav upravljanja robotom preko Wi-Fi bežične komunikacije" - cilj projekta je bio izraditi centralni sustav za upravljene jednog ili više robota preko Wi-Fi bežične komunikacije mreže
                \item \textbf{Kolegiji:} Algoritmi i strukture podataka, Baze podataka, Operacijski sustavi, Programsko inženjerstvo, Arhitektura računala
            \end{highlights}
        \end{onecolentry}

    
    \section{Tehnologije}



        
        \begin{onecolentry}
            \textbf{Programski jezici:} Python, Rust, SQL
        \end{onecolentry}

        \vspace{0.2 cm}

        \begin{onecolentry}
            \textbf{Technologies:} Spark, SQL Server, Azure Synapse, Azure Data Factory, Azure Databricks, Azure Data Lake
        \end{onecolentry}

    
    

\end{document}